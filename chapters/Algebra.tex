\chapter{Algebra}

This chapter introduces fundamental mathematical concepts of Algebra.

\section{Basic Algebraic Objects}

\subsection{Number Systems}

\begin{itemize}
    \item \textbf{Natural Numbers:} $\mathbb{N} = \{1,2,3, \dots \}$  will occasionally be expanded to contain \O
    \item \textbf{Integers:} $\mathbb{I} = \{ \dots, -2, -1, 0, 1, 2 \dots \}$
    \item \textbf{Rationals:} $\mathbb{Q} = \left\{\frac{p}{q} \mid p, q \in \mathbb{Z}, q \neq 0\right\}$
    \item \textbf{Irrationals:} are numbers that cannot be expressed as fractions like $\pi$ or $\sqrt{2}$
    \item \textbf{Real Numbers:} $\R = \Q \cup \left\{Irrational Numbers\right\}$
\end{itemize}
\subsection{Variables and Constants}

A variable is a symbol that represents a number.

A constant is a fixed number such as 4.

\subsection{Operations}

An \textbf{operation} is a rule that takes one or more elements from a given set and produces a single element of the same set.\\

\begin{definition} 

Operation

An \emph{$n$-ary operation} on a set $S$ is a function
\[
\star : S^{n} \to S
\]
that assigns one element of $S$ to every ordered $n$-tuple of elements of $S$.


\end{definition}

An operation that accepts one input is called a \emph{unary operation}. An operation that accepts two inputs is called a \emph{binary operation}, and so on.

\subsection{Algebraic Expressions}

An \textbf{algebraic expression} is a finite combination of variables, constants, and operations. Algebraic expressions do not assert equality.

\subsection{Equations}

An \textbf{equation} is a statement that two expressions are equal. A solution is a value of the variable that makes the equation true.


\section{Basic Definitions}

We begin with some basic definitions that are essential for understanding the material presented in later chapters.

\begin{definition}
A \textbf{set} is a well-defined collection of distinct objects, considered as an object in its own right.
\end{definition}

\begin{example}
The set of natural numbers is denoted by $\mathbb{N} = \{1, 2, 3, 4, \ldots\}$.
\end{example}

\section{Fundamental Algebraic Operations}

In this section, we explore some fundamental operations that are used to manipulate algebraic expressions and equations.

Let $\textit{a}, \textit{b} \in \R$.

\subsection{Addition}

Addition is denoted by $\textit{a} + \textit{b}$. Addition is a binary operation combining two numbers into their sum.

\begin{definition}

\textbf{Additive inverse} is the number which when added to another number results in the sum being \0.

\end{definition}

\begin{definition}

\textbf{Additive Identity} is the number which when added to another number results in a sum equal to that other number.
    
\end{definition}



\subsection{Subtraction}

Subtraction is denoted by $\textit{a} - \textit{b}$. Subtraction is the same as adding the additive inverse.
\subsection{Multiplication}

\subsection{Division}

\begin{theorem}
For any real numbers $a$ and $b$, if $a < b$, then $a + c < b + c$ for any real number $c$.
\end{theorem}

\begin{proof}
Let $a, b, c \in \mathbb{R}$ with $a < b$. By the properties of real numbers, adding the same value to both sides of an inequality preserves the inequality. Therefore, $a + c < b + c$.
\end{proof}

\section{Exercises}

\begin{enumerate}
    \item Prove that the sum of two even numbers is even.
    \item Show that if $a$ and $b$ are rational numbers, then $a + b$ is also rational.
    \item Give an example of an irrational number and prove that it is irrational.
\end{enumerate}
