\chapter{Trigonometry}

This chapter introduces the concepts of Trigonometry.

\section{Limits and Continuity}

The concept of a limit is fundamental to calculus.

\begin{definition}
Let $f$ be a function defined on an open interval containing $a$ (except possibly at $a$ itself). We say that the \textbf{limit} of $f(x)$ as $x$ approaches $a$ is $L$, written
\[
\lim_{x \to a} f(x) = L,
\]
if for every $\epsilon > 0$, there exists a $\delta > 0$ such that whenever $0 < |x - a| < \delta$, we have $|f(x) - L| < \epsilon$.
\end{definition}

\begin{definition}
A function $f$ is \textbf{continuous} at a point $a$ if
\[
\lim_{x \to a} f(x) = f(a).
\]
\end{definition}

\section{Derivatives}

The derivative measures the rate of change of a function.

\begin{definition}
The \textbf{derivative} of a function $f$ at a point $x$ is defined as
\[
f'(x) = \lim_{h \to 0} \frac{f(x + h) - f(x)}{h},
\]
provided this limit exists.
\end{definition}

\begin{theorem}[Chain Rule]
If $g$ is differentiable at $x$ and $f$ is differentiable at $g(x)$, then the composite function $f \circ g$ is differentiable at $x$, and
\[
(f \circ g)'(x) = f'(g(x)) \cdot g'(x).
\]
\end{theorem}

\section{Integration}

Integration is the inverse operation to differentiation.

\begin{definition}
Let $f$ be a function on $[a, b]$. The \textbf{definite integral} of $f$ from $a$ to $b$ is denoted
\[
\int_a^b f(x) \, dx.
\]
\end{definition}

\begin{theorem}[Fundamental Theorem of Calculus]
If $f$ is continuous on $[a, b]$ and $F$ is an antiderivative of $f$ on $[a, b]$, then
\[
\int_a^b f(x) \, dx = F(b) - F(a).
\]
\end{theorem}

\section{Exercises}

\begin{enumerate}
    \item Evaluate $\lim_{x \to 0} \frac{\sin x}{x}$.
    \item Find the derivative of $f(x) = x^3 \sin(2x)$.
    \item Compute $\int_0^1 x^2 e^x \, dx$ using integration by parts.
\end{enumerate}
