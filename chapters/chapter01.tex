\chapter{Introduction to Mathematical Concepts}

This chapter introduces fundamental mathematical concepts that will be used throughout the book.

\section{Basic Definitions}

We begin with some basic definitions that are essential for understanding the material presented in later chapters.

\begin{definition}
A \textbf{set} is a well-defined collection of distinct objects, considered as an object in its own right.
\end{definition}

\begin{example}
The set of natural numbers is denoted by $\mathbb{N} = \{1, 2, 3, 4, \ldots\}$.
\end{example}

\section{Fundamental Properties}

In this section, we explore some fundamental properties of mathematical objects.

\begin{theorem}
For any real numbers $a$ and $b$, if $a < b$, then $a + c < b + c$ for any real number $c$.
\end{theorem}

\begin{proof}
Let $a, b, c \in \mathbb{R}$ with $a < b$. By the properties of real numbers, adding the same value to both sides of an inequality preserves the inequality. Therefore, $a + c < b + c$.
\end{proof}

\section{Exercises}

\begin{enumerate}
    \item Prove that the sum of two even numbers is even.
    \item Show that if $a$ and $b$ are rational numbers, then $a + b$ is also rational.
    \item Give an example of an irrational number and prove that it is irrational.
\end{enumerate}
